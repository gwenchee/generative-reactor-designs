\section{Method}

\subsection{Genetic Algorithm}

\subsection{Reactor Geometry Generation}
The biggest challenge for generatively designing reactor geometries with 
non-classical shapes is defining the parameters to be varied. 
A good balance must be struck between limiting the number of degrees of freedom 
(parameters) and giving the algorithm the flexibility to create unique geometries.
Examples of projects that pursue this balance are discussed below. 
Bergmann et al \cite{bergmann_simulation_2018} described a 2D re-entrant hole 
in the Swiss Spallation Neutron Source with 9 variables that allowed the genetic 
algorithm to cover a large exploration space of CAD geometries. 
The 9 variables included five distance values and four angle values. 
Veenstra et all \cite{veenstra_evolution_2018} described the undulation of 
a Knife-fish soft robot with a fourier series. 
The genetic algorithm varied five terms in the fourier series to find the undulation 
that resulted in the fastest swimming. 
For this project, we can learn from them by creating parameters that
express reactor geometry simply while allowing for a large exploration space. 

\subsubsection{Constraints}
Reactor geometry constraints must be defined to ensure that only realistic 
geometries are accepted. 
Constraints are: 
\begin{itemize}
    \item Each cooling channel must fully go through the reactor from top to 
    bottom. This is to prevent cooling channels from not having a through 
    path, resulting in lack of heat transfer. 
\end{itemize}

\subsubsection{Parameters}

