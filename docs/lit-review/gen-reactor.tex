\section{Generative Reactor Design}
For nuclear reactors, generative design is constrained by 
variables such as mass or volume of fuel, fuel enrichment, effectiveness 
of heat transfer, effective neutron multiplication factor, etc. 
Nuclear reactors not only experience physical forces, but also
require evaluation of the neutronics of the system, therefore generative design of 
a nuclear reactor cannot make use of tools such as Autodesk Fusion360 or SolidWorks 
Topology. 
Therefore, a framework that couples well-developed advanced genetic algorithms 
with well-supported monte-carlo particle transport 
codes (Serpent \cite{leppanen_serpent_2014}, 
MCNP \cite{werner_mcnp6._2018}, etc.) and thermal hydraulics 
codes (RELAP7 \cite{andrs_relap-7_2012} etc.)
must be created to successfully produce generative reactor designs. 

\subsection{Workflow}
% How the genetic algorithm is coupled with a CAD designer and analysis tool 
% (serpent/RELAP)
% CAD designer > grasshopper3d (see Bryne Paper)
Figure \ref{fig:workflow} depicts a framework for leveraging genetic algorithms
to design nuclear reactors. 
This is a general framework that does not specify algorithms or analytical 
softwares. 
Instead, it provides placeholders for algorithm and software types, so 
that a user may choose the types of algorithms and softwares they want to use in 
their mission towards using generative design to design a nuclear reactor. 
Table \ref{tab:examples} lists algorithm and software types that could be used 
in the framework. 

\begin{figure}[!htbp]
        \centering
        \begin{tikzpicture}[node distance=2cm]
                \tikzstyle{every node}=[font=\small]
                \node[anchor=west] (1) [noblock] {\textbf{Create}};
                \node (2) [bblock, right=of 1] {\textbf{GA: Creates population}};
                \node (3) [pblock, right=of 2] {\textbf{GGS: Generates reactor designs}};
                \node[anchor=west] (4) [noblock, below of= 1] {\textbf{Analyze}};
                \node (5) [oblock, below of =2]{\textbf{NT: Run NT code for each solution}};
                \node (6) [gblock, right=of 5]{\textbf{TH: Run TH code for each solution}};
                \node[anchor=west] (7) [noblock, below of= 4] {\textbf{Evaluate}};
                \node (8) [bblock, below of =5] {\textbf{GA: Evaluate population}};
                \node (9) [ppblock, right=of 8] {\textbf{M: Evaluation based on performance metrics}};
                \node[anchor=west] (10) [noblock, below of= 7] {\textbf{Check}};
                \node (11) [lppblock, below of=8, xshift=3cm, yshift=-0.2cm] {\textbf{M: Is termination criteria met?}};
                \node (16) [lbblock, below of = 11] {\textbf{GA: Best solution is returned!}};
                \node[anchor=west] (12) [snoblock, above of= 2, xshift=3cm, yshift=-1.1cm] {};
                \node[anchor=west] (13) [snoblock, below of=12,yshift=-0.15cm] {};
                \node[anchor=west] (14) [snoblock, below of=13,yshift=-0.02cm] {};
                \node[anchor=west] (15) [snoblock, below of=14,yshift=-0.02cm] {};
                \node[anchor=west] (17) [noblock, below of= 10] {\textbf{Completed}};
                \draw [-,thick] (11) -- ([shift={(3cm,0cm)}]11.east) |- node[anchor=west, yshift=-4.2cm] {no} ([shift={(0cm,0.7cm)}]12.north)--(12);
                \draw [arrow] (12) -- ([shift={(0cm,0cm)}]12.north) |- ([shift={(0cm,0.4cm)}]2.north)--(2);
                \draw [arrow] (12) -- ([shift={(0cm,0cm)}]12.north) |- ([shift={(0cm,0.4cm)}]3.north)--(3);
                \draw [-,thick] (2) -- ([shift={(0cm,0cm)}]2.south) |- ([shift={(0cm,0.3cm)}]13.north)--(13);
                \draw [-,thick] (3) -- ([shift={(0cm,0cm)}]3.south) |- ([shift={(0cm,0.3cm)}]13.north)--(13);
                \draw [arrow] (13) -- ([shift={(0cm,0cm)}]13.north) |- ([shift={(0cm,0.3cm)}]5.north)--(5);
                \draw [arrow] (13) -- ([shift={(0cm,0cm)}]13.north) |- ([shift={(0cm,0.3cm)}]6.north)--(6);
                \draw [-,thick] (5) -- ([shift={(0cm,0cm)}]5.south) |- ([shift={(0cm,0.3cm)}]14.north)--(14);
                \draw [-,thick] (6) -- ([shift={(0cm,0cm)}]6.south) |- ([shift={(0cm,0.3cm)}]14.north)--(14);
                \draw [arrow] (14) -- ([shift={(0cm,0cm)}]14.north) |- ([shift={(0cm,0.3cm)}]8.north)--(8);
                \draw [arrow] (14) -- ([shift={(0cm,0cm)}]14.north) |- ([shift={(0cm,0.21cm)}]9.north)--(9);
                \draw [-,thick] (8) -- ([shift={(0cm,0cm)}]8.south) |- ([shift={(0cm,0.3cm)}]15.north)--(15);
                \draw [-,thick] (9) -- ([shift={(0cm,0cm)}]9.south) |- ([shift={(0cm,0.3cm)}]15.north)--(15);
                \draw [arrow] (15) -- ([shift={(0cm,0cm)}]15.north) |- ([shift={(0cm,0.1cm)}]11.north)--(11);
                \draw [arrow] (11) -- ([shift={(0cm,0cm)}]11.south) |- node[anchor=west, yshift=0.3cm] {yes} ([shift={(0cm,0.1cm)}]16.north)--(16);
                \matrix [draw,below left,yshift=-0.5cm, xshift=-7cm] at (current bounding box.south east) {
                \node [bbblock,label=right:\textbf{GA: Genetic algorithm}] {}; \\
                \node [bpblock,label=right:\textbf{GGS: Geometry generating software}] {}; \\
                \node [boblock,label=right:\textbf{NT: Neutron transport code}] {}; \\
                \node [bgblock,label=right:\textbf{TH: Thermal hydaulics code}] {}; \\
                \node [bppblock,label=right:\textbf{M: User-defined metrics/criteria}] {}; \\
                };
        \end{tikzpicture}
        \caption{Generative reactor design framework. Each component of the 
        framework is user-selected.}
        \label{fig:workflow}
\end{figure}

\begin{table}[!htbp]
        \caption{Example algorithm and software types to populate generative reactor 
        design framework (Fig \ref{fig:workflow}).}
        \label{tab:examples}
        \centering
        \doublespacing
        \small
        \begin{tabular}{lp{10cm}}
        \hline
        \textbf{Framework Component} & \textbf{Examples}\\ \hline
        Genetic algorithm driver & DEAP \cite{fortin_deap_2012} \\
        Geometry generating software & Trelis \cite{noauthor_trelis_2018}, FreeCAD \cite{falck_freecad_2012}, SolidWorks \cite{lombard_solidworks_2008}, grasshopper3d \cite{rutten_grasshopper3d_2015}, GenerativeComponents \cite{aish_bentleys_2003}, Trilinos \cite{heroux_overview_2003} \\
        Neutron transport code & Serpent \cite{leppanen_serpent_2014}, MCNP \cite{werner_mcnp6._2018}, SCALE \cite{bucholz_scale:_1982}, OpenMC \cite{romano_openmc_2013} \\ 
        Thermal hydraulics code & RELAP5-3D \cite{strydom_comparison_2016}, RELAP7 \cite{andrs_relap-7_2012}, TRACE \cite{xu_multi-physics_2006}\\
        User-defined metrics/criteria & keff, heat transfer rate, fuel enrichment, mass of fuel \\ \hline
\end{tabular}
\end{table}

\subsection{Software}
The generative reactor design framework proposed (Fig. \ref{fig:workflow}) 
has many coupled components. 
To minimize the amount of coupling required, software is chosen with 
framework-leanness in mind. 
The shortlisted software will be evaluated based on the clearly 
defined requirements of each component. 
Each iteration of the framework requires the neutron transport code, 
the thermal hydraulics code, and evaluation to be run for each 
nuclear reactor geometry (individual) generated by the genetic algorithm. 
Therefore, to ensure efficiency of the framework, the individual runs 
must be parallelized. 
The generative reactor design framework is computation-heavy. 
To produce results in a timely manner, a high performance computer 
such as Blue Waters \cite{ncsa_about_2017} will be used. 
In summary, each component's software is chosen based on these 
conditions: 
\begin{enumerate}
    \item It best meets the clearly defined requirements 
    its respective component. 
    \item It is parallelizable. 
    \item It is compatible with a high performance computer. 
    \item It is open-source and free (preferable but not required). 
\end{enumerate}

The subsequent subsections will define the requirements of each 
component in the generative reactor framework, the software considered, 
and an explanation justifying the final software choice. 

\subsubsection{Genetic Algorithm Driver}
Evolutionary algorithm computation is a sophisticated field with diverse
techniques and mechanisms, resulting in even well designed frameworks 
being complicated under the hood \cite{fortin_deap_2012}. 
These implementation intricacies are difficult to extend since it requires 
the user to edit the source code. 
This is concerning since using evolutionary algorithms to solve unique 
real-world problems requires customization of algorithms 
\cite{fortin_deap_2012}. 
Therefore, an evolutionary algorithm computation framework that gives 
the user the capability to build custom evolutionary algorithms is 
required for this project. 

There are many evolutionary algorithm computation packages 
available: \gls{DEAP} \cite{fortin_deap_2012}, 
inspyred \cite{garrett_inspyred_2014}, 
Pyevolve \cite{perone_pyevolve_2009}, and
OpenBEAGLE \cite{gagne_open_2002}. 
DEAP is the most newly created package and places a high value on 
code compactness and code clarity \cite{fortin_deap_2012}. 
A comparison of the number of lines of code each package requires 
to define different components of a test problem, and \gls{DEAP} 
was the only framework that completely defined the test problem 
in less than one hundred lines of code \cite{fortin_deap_2012}.
\gls{DEAP} is the only framework that allows the user to rapidly 
prototype evolutionary algorithms and define custom algorithms without 
having to dig deep into the source code to modify lines, making it 
the code of choice for the genetic algorithm driver component of 
the generative reactor design framework. 

\subsubsection{Geometry Generating Software}


\subsubsection{Neutron Transport Code}

\subsubsection{Thermal Hydraulics Code}

\subsection{Evaluation Metrics}

