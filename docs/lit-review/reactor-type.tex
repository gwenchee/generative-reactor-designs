\section{\gls{FHR}}
For this project, generative reactor design is explored for the 
\gls{FHR}. 
The \gls{FHR} is a reactor concept introduced in 2012 that uses high-temperature 
coated-particle fuel and a low pressure liquid fluoride-salt coolant 
\cite{forsberg_fluoride-salt-cooled_2012,facilitators_fluoride-salt-cooled_2013}.
\gls{FHR} technology combines the best aspects of \gls{MSR} and\gls{HTGR} 
technologies. 
Molten fluoride salts as working fluids for nuclear reactors have been explored 
since the 1960s and are desirable because of the salts' high-temperature 
performance and overall chemical stability \cite{scarlat_design_2014}.  
Using molten salts for reactor coolant introduces inherent safety compared 
to water due to the salts' high boiling temperature and high volumetric 
heat capacity, eliminating the risk of coolant boiling off, resulting in 
fuel elements overheating \cite{ho_molten_2013}. 
The leading candidate coolant salt is the fluoride salt Li$_2$BeF$_4$ (FLiBe), 
which remains liquid without pressurization up to 1400$\degree$C and a larger 
$\rho C_p$ than water \cite{ho_molten_2013,forsberg_fluoride-salt-cooled_2012}. 
\glspl{FHR} are favorable compared to a liquid fuel reactor because the solid 
fuel cladding adds an extra barrier to fission product release 
\cite{ho_molten_2013}.
\gls{HTGR} technology has been studied since the 1970s because it delivers 
heat at substantially high temperatures than \glspl{LWR} resulting in 
the following advantages: increased power conversion efficiency, reduced 
waste heat generation, and co-generation and process heat capabilities 
\cite{scarlat_design_2014}. 
In \glspl{HTGR}, the helium coolant is held at a high pressure of approximately 
100 atm, while \gls{FHR}'s FLiBe coolant is at room pressure, resulting in lower 
construction costs since a thick concrete reactor vessel is not required.
The molten salt coolant has superior cooling and moderating properties compared 
to helium coolant in \glspl{HTGR}, resulting in \glspl{FHR} operating at 
power densities two to six times higher than  \glspl{HTGR} 
\cite{scarlat_design_2014,forsberg_fluoride-salt-cooled_2012}.
Therefore, by combining the FLiBE coolant from \gls{MSR} technology and 
\gls{TRISO} particles from \gls{HTGR} technology, the \gls{FHR} benefits from 
the low operating pressure and large thermal margin provided by using a molten 
salt coolant and the accident-tolerant qualities of \gls{TRISO} particle fuel. 

There are several types of \gls{FHR} designs that are being developed worldwide: 
circulating pebbles (\gls{PBFHR}, Kairos Power Reactor) and hexagonal fuel plate elements (\gls{AHTR}). 

% Benchmark? 

\subsection{Modelling Challenges}
% why is it so hard to model a FHR
Verification and validation of a simulation tool is a crucial step to successfully 
model a reactor with the goal of deployment \cite{rahnema_phenomena_2019}. 
The \gls{FHR} has a complex core design due to the multiple heterogeneity present 
in the fuel introduced by presence of \gls{TRISO} particles and other moderating 
materials (i.e.in the \gls{AHTR}, \gls{TRISO} particles pressed into plates, which 
are bundled into assembles) \cite{ramey_monte_2018,rahnema_phenomena_2019}.
Traditional homogenization methods are insufficient to capture the correct physics 
in \glspl{FHR}, due to the multiple heterogeneity \cite{ramey_monte_2018}. 
For example, in the \gls{PBFHR}, pebble packing variation effects are significant, 
resulting in large errors at the outer edge of the core when homogenized 
\cite{rahnema_phenomena_2019}. 
In the \gls{AHTR}, single and multiple slab homogenization decreased computation time 
by 10, however they introduce a nontrivial error of $\sim$3\%
\cite{ramey_monte_2018,cisneros_neutronics_2012}.
% random stuff to take note of 
Both full core peaking factor and within-assembly peaking factor must be evaluated, 
due to the non-uniformity in fission density of individual \gls{TRISO} particles. 
Neutron spectra in different materials and regions must be investigated and compared 
against assembly averaged spectra. 
The \gls{FHR} consists of materials that are known to have uncertainty 
in nuclear cross-section data such as the moderation, thermalization, and absorption 
for FliBE constituents and the thermalization and absorption 
in the carbon within the graphite matrix \cite{rahnema_phenomena_2019}. 
 
\subsection{Previous \gls{FHR} Modeling and Simulation Efforts}
% Previous efforts to model each type of FHR. Plate & Pebble  
Ramey et al \cite{ramey_monte_2018} conducted a 2D \gls{AHTR} assembly model with 
continuous energy cross sections in SERPENT (\gls{TRISO}-level results were also 
included). 
Zhang et al introduced a 3D \gls{AHTR} benchmark \cite{zhang_stylized_2018}
and solved it using the continuous-energy COMET code 
\cite{zhang_continuous-energy_2018}. 
The COMET solution was in good agreement with the \gls{MCNP} reference solutions
\cite{zhang_continuous-energy_2018}. 



